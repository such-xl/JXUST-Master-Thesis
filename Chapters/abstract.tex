%摘要开始部分
\setcounter{page}{0}						% 设置当前页页码编号从1开始计数
\pagenumbering{Roman}						% 设置页码字体为小写阿拉伯字体
\fancyhead[C]{江西理工大学硕士学位论文\quad 摘要} % 设置学院
\renewcommand{\headrulewidth}{0.5pt} %设置页眉横线粗细
\addcontentsline{toc}{section}{摘要}\tolerance=500 %将摘要放进目录
\section*{\centering \hei \zihao{-3} 摘\quad 要}
\song \zihao{-4} \setstretch{1.523} %1.523是22pt
多智能体强化学习(MARL)因其分散式决策能力和良好适应性,已成为解决动态柔性作业车间调度问题(DFJSP)的有前景方法。然而,MARL 在 DFJSP 中仍面临由多智能体交互引发的环境非稳态、奖励稀疏以及策略收敛缓慢等挑战。为此,本文提出了一种面向机器对齐的 MARL 框架,通过利用优先级派工规则(Priority Dispatching Rules, PDRs)构建抽象化的状态—动作表示,将不同类型机器的决策逻辑统一,从而使单一策略即可协调多智能体的调度行为,有效缓解因智能体交互产生的非稳态问题。在此基础上,本文引入在线专家机制,通过实时生成专家动作对智能体策略进行引导,使其在训练早期即可形成优良的决策倾向,加快收敛并提升策略稳定性。同时,结合好奇心机制构建内在奖励,使智能体在奖励稀疏或延迟的情况下仍能主动探索关键调度状态,从而增强策略的泛化能力与环境适应性。实验结果表明,所提方法在多智能体 DFJSP 环境下能够显著降低作业延迟,提高调度效率,并表现出良好的策略协调性与学习稳定性。

\par 
\mbox{}
\par 
\hei \textbf{关键字:} \song 关键字1,关键字2,关键字3

\newpage
\clearpage
\addcontentsline{toc}{section}{Abstract}\tolerance=500 %将摘要放进目录 
\section*{\centering \zihao{3} Abstract}
\zihao{-4} \setstretch{1.384} 

dd
\par
\mbox{}
\par 
\textbf{Keywords:} keyword1, keyword2, keyword3